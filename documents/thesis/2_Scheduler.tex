\section{Verifying a Basic Eio Scheduler}
\label{sec:scheduler}

Cooperative concurrency schedulers for user-level threads (i.e. \emph{fibers}) are commonly treated in the literature
on effect handlers~\cite{dolan2018concurrent,leijen2017structured,de2021separation} because they are a good example for the usefulness
of manipulating delimited continuations with effect handlers.
Generally, the scheduler contains an effect handler and fibers are normal functions which perform effects to yield execution.
Performing an effect causes execution to jump to the enclosing effect handler, providing it with the rest of the fiber's computation in the form of a delimited continuation.
The scheduler keeps track of a collection of these continuations and by invoking one of them it can schedule the next fiber.
This approach is also used in Eio.

In the following we define a basic model of the Eio scheduler and related data structures such as promises.
Throughout the thesis we then extend this model with more features.
We first discuss the implementation of our model and give an intuition about the behavior of each component in section~\ref{sec:sched-impl}.
Based on this intuition we then build a formalization in section~\ref{sec:sched-spec}.

\subsection{Implementation}
\label{sec:sched-impl}

Let us first get an idea of how the different core elements of Eio interact by looking at their types.
The code we present throughout the thesis is \ocf{} code that represents the \hh{} code we verify.

\begin{minted}{ocaml}
  (* Basic interface of the Eio library. *)
  Scheduler.run : (unit -> 'a) -> 'a
  Fiber.fork_promise : (unit -> 'a) -> 'a Promise.t
  Promise.await : 'a Promise.t -> 'a
\end{minted}

\ocamlin{Scheduler.run} is the main entry point to Eio.
It runs a scheduler and is provided a function which represents main fiber.
A scheduler runs the main fiber and all forked off fibers in a single thread, but we already assume that fibers can run in different threads to build key data structures in a thread-safe way.

The \ocamlin{Fiber.fork_promise} function is used to fork off fibers in the current scheduler.
The function returns a promise holding the eventual return value of the new fiber.
The promise is thread-safe so that it can be shared with fibers running in different threads.
The \ocamlin{Promise.await} function can be used by any fiber to suspend execution until the value of a promise is available.
Common problems like deadlocks are not prevented in any way and are the responsibility of the programmer.

\subsubsection{\ocamlin{Scheduler.run}}
\label{sec:sched-impl-run}

\begin{figure}[ht]
  \begin{minted}[escapeinside=@@]{ocaml}
module Scheduler = struct
  type _ eff += Fork : (unit -> unit) -> unit eff
  type 'a waker : 'a -> unit
  type _ eff += Suspend : ('a waker -> unit) -> 'a eff

  let run init (main: unit -> 'a) : 'a =
    let result = ref None in
    let run_queue = Queue.create () in  @\label{ln:run_queue}@
    let rec next () =  @\label{ln:run_next}@
      match Queue.pop run_queue with
      | None -> begin
          match !result with @\label{ln:run_check}@ 
          | None -> next ()
          | Some _ -> ()
        end
      | Some fiber -> fiber () 
    in
    let rec execute fiber = @\label{ln:run_execute}@ 
      match fiber () with
      | () -> next ()  @\label{ln:run_value}@
      | @\texttt{\textbf{effect}}@ (Fork fiber), k ->  @\label{ln:run_fork}@
          Queue.push run_queue (fun () -> continue k ()); @\label{ln:run_push}@ 
          execute fiber @\label{ln:run_rec}@
      | @\texttt{\textbf{effect}}@ (Suspend register), k ->  @\label{ln:run_suspend}@
          let waker = fun v -> Queue.push run_queue (fun () -> continue k v) in @\label{ln:run_waker}@  
          register waker;  @\label{ln:run_register}@
          next ()
    in
    let tlv = ref init in
    execute (fun () -> result := Some (main ())); @\label{ln:run_main}@ 
    match !result with @\label{ln:run_return}@
    | None -> error "impossible"  @\label{ln:run_error}@
    | Some result -> result
end
  \end{minted}
  \caption{Implementation of \ocamlin{Scheduler.run}.}
  \label{fig:sched-impl-run}
\end{figure}

As mentioned above this is the main entry point to the Eio library and its code is shown in figure~\ref{fig:sched-impl-run}.
It sets up the scheduler environment and then runs the main fiber (and every subsequent fiber) under an effect handler.

The \ocamlin{result} reference eventually holds the final value of the main fiber.
The \ocamlin{run_queue} (line~\ref{ln:run_queue}) contains closures that invoke the continuation of an effect.
The closures represent ready fibers which can continue execution from the point where they performed an effect.
The \ocamlin{next} function (line~\ref{ln:run_next}) pops one fiber from the \ocamlin{run_queue} and executes it.
If no more ready fibers remain, the function will check if the main fiber has already finished execution~\ref{ln:run_check} and if so it will also exit, which causes the scheduler to return the main fiber's final value (line~\ref{ln:run_return}).
Otherwise, the main fiber's continuation exists \emph{somewhere} -- it could be deadlocked or just awaiting a promise from a different thread -- so the \ocamlin{next} function busy loops until a fiber becomes available again.
Busy looping makes sense in this case because other threads can push values into the \ocamlin{run_queue}.
For the verification we assume the specification of a suitable \ocamlin{Queue} module that supports thread-safe push and pop operations and given a predicate \(I\), maintains that all elements \(v\) in the queue satisfy \(I~v\).
The inner \ocamlin{execute} function (line~\ref{ln:run_execute}) is called once on each fiber to evaluate it and handle any performed effects.

\subsubsection*{Value Case}
The non-effect case of the \ocamlin{match} (line~\ref{ln:run_value}) only runs the next fiber because Eio adopts the convention that all fibers return a unit value and their real return value is handled out of band.
\begin{itemize}
  \item The main fiber is wrapped in a closure that saves its return value in a reference (line~\ref{ln:run_main}).
  \item All other fibers are forked using \ocamlin{Fiber.fork_promise}, which wraps them in a closure that saves their return value in a promise.
\end{itemize}
%
This emphasizes the fact that an Eio scheduler is only used for running fibers.
The interaction between fibers waiting for values of other fibers is handled separately by promises.

\subsubsection*{\efork{} Case}
Handling a \efork{} effect (line~\ref{ln:run_fork}) is simple because it only carries a new \ocamlin{fiber} to be executed, so the handler recursively calls \ocamlin{execute} (line~\ref{ln:run_rec}) on it.
The execution of the original fiber is paused due to performing an effect and its continuation \ocamlin{k} is placed in the run queue so that it can be scheduled again (line~\ref{ln:run_push}).
This prioritizes the execution of a new fiber and is a design decision by Eio.
It would be equally valid to push the closure \ocamlin{(fun () -> execute fiber)} into the run queue instead, to give priority to the already running fiber.

\subsubsection*{\esuspend{} Case}
Handling a \esuspend{} effect (line~\ref{ln:run_suspend}) may look complicated at first due to the higher-order \ocamlin{register} function.
This effect is used by fibers to suspend execution until a condition is met.
The fiber defines this condition by constructing a \ocamlin{register} function which in turn receives a wake-up capability by the scheduler in the form of a \ocamlin{waker} function.
The key point is that as long as the continuation \ocamlin{k} is not invoked, the fiber does not continue execution.
So the \ocamlin{waker} function "wakes up" a fiber by placing its continuation \ocamlin{k} into the run queue (line~\ref{ln:run_waker}).
The \ocamlin{register} function is called by the scheduler right after the fiber suspends execution (line~\ref{ln:run_register}) and is responsible for installing \ocamlin{waker} as a callback at a suitable place (or even call it directly).
For example, to implement awaiting promises, the \ocamlin{waker} function is saved in a data structure that calls the function after the promise is fulfilled.

Note that the \ocamlin{waker} function's argument \ocamlin{v} has a \emph{locally abstract type}, which is a typical pattern in effect handlers.
From the point of view of the fiber, the polymorphic type \ocamlin{'a} of the \esuspend{} effect is instantiated depending on how the effect's return value is used.
But the scheduler does not get any information about this so the argument type of the continuation \ocamlin{k} and the \ocamlin{waker} function is abstract.

% <!-- In our simplified model of Eio, the `Suspend` effect is only performed in the implementation of `Promise.await` which registers the `waker` callback to be called when the promise is fulfilled. -->

\subsubsection{\ocamlin{Fiber.fork_promise}}
\label{sec:sched-impl-fork}

\begin{figure}[ht]
  \begin{minted}[escapeinside=@@]{ocaml}
module Promise = struct
  type 'a state = Done of 'a | Waiting of Broadcast.t
  type 'a t = 'a state Atomic.t

  let create () : 'a t = 
    let bcst = Broadcast.create () in
    Atomic.create (Waiting bcst)

  let fulfill (p: 'a t) (result: 'a) = 
    match Atomic.get p with
    | Done _ -> error "impossible" @\label{ln:fork_error}@ 
    | Waiting bcst ->
        Atomic.set p (Done result);
        Broadcast.signal_all bcst @\label{ln:fork_signal}@ 

  (* ... *)
end
  
module Fiber = struct
  let fork_promise (f: unit -> 'a) : 'a Promise.t =
    let p = Promise.create () in @\label{ln:fork_create}@ 
    let fiber = fun () ->
      let result = f () in
      Promise.fulfill p result @\label{ln:fork_fulfill}@ 
    in
    perform (Fork fiber); @\label{ln:fork_perform}@
    p
end
  \end{minted}
  \caption{Excerpt of the \ocamlin{Promise} module \& implementation of \ocamlin{Fiber.fork_promise}.}
  \label{fig:sched-impl-fork}
\end{figure}

This function is the basic way to fork a new fiber in Eio and the only one we model in our development.
The code is presented in figure~\ref{fig:sched-impl-fork}.
It creates a promise (line~\ref{ln:fork_create}) and spawns the provided function as a new fiber using the \efork{} effect (line~\ref{ln:fork_perform}).
Promises are always created in a \ocamlin{Waiting} state (we also say \emph{unfulfilled}) and calling \ocamlin{Promise.fulfill} sets it to the \ocamlin{Done} state, at which point the final value can be retrieved.
When \ocamlin{f ()} is reduced to a value \ocamlin{result}, the promise is fulfilled with that value (line~\ref{ln:fork_fulfill}), which signals all fibers waiting for that result to wake up (line~\ref{ln:fork_signal}).
The meaning of the \ocamlin{Broadcast.t} contained in a promise is explained in the next section.

\subsubsection{\ocamlin{Promise.await}}
\label{sec:sched-impl-await}

\begin{figure}[ht]
  \begin{minted}[escapeinside=@@]{ocaml}
module Promise = struct
  let make_register (p: 'a t) (bcst: Broadcast.t) : (unit waker -> unit) =
    fun waker ->
      let register_result = Broadcast.register bcst waker in @\label{ln:await_register}@
      match register_result with
      | Invoked -> ()
      | Registered register_handle ->
        match Atomic.get p with @\label{ln:await_match3}@
        | Done result ->  
            if Broadcast.try_unregister register_handle
            then waker () @\label{ln:await_waker}@
            else ()
        | Waiting _ -> ()

  let await (p: 'a t) : 'a =
    match Atomic.get p with @\label{ln:await_match1}@
    | Done result -> result
    | Waiting bcst -> begin @\label{ln:await_waiting1}@
        let register = make_register p bcst in
        perform (Suspend register); @\label{ln:await_perform}@
        match Atomic.get p with @\label{ln:await_match2}@
        | Done result -> result 
        | Waiting _ -> error "impossible" @\label{ln:await_error}@
      end
    
  (* ... *)
end
  \end{minted}
  \caption{Implementation of \ocamlin{Promise.await}.}
  \label{fig:sched-impl-await}
\end{figure}

This is the most complex looking function in our development which is partly due to the \esuspend{} effect and also due to the use of \emph{broadcast} functions.
Its code is presented in figure~\ref{fig:sched-impl-await}.
The purpose of \ocamlin{Promise.await p} is to suspend execution of the calling fiber until \ocamlin{p} is fulfilled with a value and then return this value.
The "suspend execution" part is handled by performing a \esuspend{} effect.
Then, the "until \ocamlin{p} is fulfilled" part is implemented by using the \emph{broadcast} data structure.

In Eio, a broadcast is an implementation of a signalling mechanism used for similar purposes as condition variables in various languages.
The major differences are that a broadcast does not use a mutex (it is a \emph{lock-free} data structure) and that callers do not directly suspend execution if the condition is not met, but supply a callback that will be called when the condition is signalled.

In figure~\ref{fig:sched-impl-broadcast} we show the public API of Eio's \ocamlin{Broadcast} module.
The \ocamlin{Broadcast.register} function attempts to register a given \ocamlin{callback} with the data structure while \ocamlin{Broadcast.signal_all} calls all registered callbacks.
For \ocamlin{Broadcast.register}, a return value of \ocamlin{Invoked} means that it already called the supplied callback because the function detected the signal while it was running.
Otherwise, a return value of \ocamlin{Registered} means that the callback was registered.
A registered callback can be unregistered by calling \ocamlin{Broadcast.try_unregister}, which returns a boolean indicating the cancellation status.
If the cancellation was successful, the previously registered callback is not called when \ocamlin{Broadcast.signal_all} is executed.
The specifications of the functions is explained in more detail in section~\ref{sec:broadcast-spec}, for now we just explain their usage in the context of \ocamlin{Promise.await}.

\begin{figure}[ht]
  \begin{minted}{ocaml}
type t
type callback = unit -> unit
type register_handle
type register_result = Invoked | Registered of register_handle

val create : unit -> t
val register : t -> callback -> register_result
val try_unregister : register_handle -> bool
val signal_all : t -> unit
  \end{minted}
  \caption{Interface of the \ocamlin{Broadcast} module.}
  \label{fig:sched-impl-broadcast}
\end{figure}

In the \ocamlin{Promise.await} function if the promise is not fulfilled initially (figure~\ref{fig:sched-impl-await} line~\ref{ln:await_waiting1})
then the fiber should wait until that is the case, so it performs a \esuspend{} effect (line~\ref{ln:await_perform}).
The \ocamlin{register} function passed to the effect registers the \ocamlin{waker} function using \ocamlin{Broadcast.register} (line~\ref{ln:await_register}).
When at some point the \ocamlin{Broadcast.signal_all} function is called -- this happens in \ocamlin{Fiber.fork_promise} -- all registered \ocamlin{waker}s are called in turn.
Recall that calling a \ocamlin{waker} function enqueues the fiber that performed the \esuspend{} effect in the scheduler's run queue so that it can continue execution.

In the default case the following simplified chain of events happens:
\begin{enumerate}
  \item The fiber suspends execution at the point of evaluating \ocamlin{perform (Suspend register)}.
  \item The \ocamlin{waker} function is registered with a broadcast.
  \item The promise is fulfilled.
  \item The \ocamlin{waker} function is called.
  \item The fiber resumes execution at the point of evaluating \ocamlin{perform (Suspend register)}.
\end{enumerate}
Therefore, when matching on the promise state again after the \esuspend{} effect returns (line~\ref{ln:await_match2}) we know the state of the promise is \ocamlin{Done} and the final value can be returned.

But because broadcast is a lock-free data structure and promises can be shared between different threads there are a number of possible interleavings that the \ocamlin{register} function must take care of as well.
The definition of the register function is interesting enough that we split it out into \ocamlin{make_register} and give a separate specification, which is not part of the public API of the module.
First, there could be a race on the state of the promise itself.
Right after the state is read (figure~\ref{fig:sched-impl-await} line~\ref{ln:await_match1}) another thread might change the state to \ocamlin{Done} and go on to call \ocamlin{Broadcast.signal_all}.
If that happens there is another possible race between the call to \ocamlin{Broadcast.register} (line~\ref{ln:await_register}) and the call to \ocamlin{Broadcast.signal_all} in the other thread\footnote{They both race to set an atomic reference holding the state of the callback registration. For more details see the implementation linked in~\cite{koval2023cqs}.}.
If \ocamlin{Broadcast.register} detects that it lost the race, it directly calls the \ocamlin{waker} function and returns \ocamlin{Invoked}.
Otherwise, the \ocamlin{waker} function is registered but in fact the \ocamlin{Broadcast.signal_all} might have already finished before \ocamlin{Broadcast.register} even started, so it failed to detect the race.
In this case the \ocamlin{waker} would be "lost" in the broadcast, never to be called.
To avoid this, \ocamlin{register} must check the state of the promise again (line~\ref{ln:await_match3}), and -- if it is fulfilled -- try to cancel the \ocamlin{waker} registration.
The cancellation fails if the \ocamlin{waker} function was already called.
Otherwise, the cancellation succeeds and the \ocamlin{register} function has the responsibility of calling \ocamlin{waker} itself (line~\ref{ln:await_waker}).

\subsubsection{Safety of the Implementation}
The \textbf{safety} concerns in the above implementation are
\begin{enumerate}
  \item \ocamlin{Scheduler.run} expecting the \ocamlin{result} reference to hold \ocamlin{Some} value after \ocamlin{execute} returns (figure~\ref{fig:sched-impl-run} line~\ref{ln:run_error})
  \item \ocamlin{Fiber.fork_promise} expecting the promise to be unfulfilled after the fiber has finished execution (figure~\ref{fig:sched-impl-fork} line~\ref{ln:fork_error}),
  \item and \ocamlin{Promise.await} expecting the promise to be fulfilled in the last match (figure~\ref{fig:sched-impl-await} line~\ref{ln:await_error}).
\end{enumerate}
In all cases, the program would crash (signified by the \ocamlin{error} expression) if the expectation is violated.
So to establish the safety of Eio we wish to prove that the expectations always hold, and the \ocamlin{error} expressions are never reached.
In the next section we show how the first two situations are addressed by defining a resource describing a one-shot assignment to a reference, and the last is a consequence of the protocol of the \esuspend{} effect.

\subsection{Specification}
\label{sec:sched-spec}

To prove specifications for an effectful program using \hazel{}, in addition defining to ghost state constructs for describing the program state space, we also need to define protocols that describe the behavior of the program's effects.
For our Eio development we modify the ghost state and the effect protocols from the cooperative concurrency scheduler development from chapter 4 of de Vilhena's dissertation~\cite{de2022proof}.

\subsubsection{Protocols}
\label{sec:sched-spec-protocols}

The protocols for the \efork{} and \esuspend{} effect are shown in figure~\ref{fig:coop-protocol-1}.
The subscripts on definitions indicate that we will change them later when extending the model.

% To have multiple (nested) areas of alignment.
% https://tex.stackexchange.com/a/586487
\begin{figure}[ht]
  \begin{align*}
    \gsIsWaker{}\; wkr\; W & \triangleq \forall v. \;  W\; v \wand{} \ewp{wkr\; v}{\bot}{\top}                                                                         \\
    \gsIsReg{1}\; reg\; W  & \triangleq \forall wkr. \; \gsIsWaker{}\; wkr\; W \wand{} \ewp{reg\; wkr}{\bot}{\top}                                                     \\
    \proto{1}              & \triangleq \begin{aligned}[t]
                                          Fork\;    & \# \; !\; e\; (e)\; \left\{\later \ewp{e\ \ ()}{\proto{1}}{\top}\right\}. ?\; ()\; \{ \top \}         \\
                                          Suspend\; & \# \; !\; reg\; W\; (reg)\; \left\{\gsIsReg{1}\; reg\; W\right\}.?\; y\; (y)\; \left\{ W\; y \right\}
                                        \end{aligned}
  \end{align*}
  \caption{Definition of \proto{1} protocol with \efork{} \& \esuspend{} effects.}
  \label{fig:coop-protocol-1}\label{spec:suspend}\label{spec:is_waker}
\end{figure}

\paragraph*{\efork{}}
The \efork{} effect accepts a value \(e\) which represents the computation that a new fiber executes.
To perform the effect one must prove that \(e\) acts as a function that can be called on unit and obeys the \proto{1} protocol itself.
This means all forked off fibers can again perform \efork{} and \esuspend{} effects.
The weakest precondition argument is guarded behind a later modality because of the recursive occurrence of \proto{1}.
% Since promise handling is done entirely in the fibers and the \efork{} effect is only used to give the fiber to the scheduler, the protocol is simplified in two ways compared to the original from the case study of de Vilhena.
% First, the scheduler does not interact with the return value of the fiber, so the weakest precondition has a trivial postcondition.
% Second, because the scheduler does not create and return the promise, the protocol itself also has a trivial postcondition.

\paragraph*{\esuspend{}}
From the type of the \esuspend{} effect in figure~\ref{fig:sched-impl-run} we already know that a value (of type \ocamlin{'a}) can be transmitted from the party that calls the \ocamlin{waker} function to the fiber that performed the effect.
The \esuspend{} protocol now expresses the same idea on the level of resources.
To suspend, a fiber must supply a function \ocamlin{register} that satisfies the \gsIsReg{1} predicate.
This predicate expresses that \ocamlin{register} can be called on a \ocamlin{waker} function for which we get to assume that it is callable on any value \(v\) that satisfies \(W\; v\).

Both \ocamlin{register} and \ocamlin{waker} must not perform effects and are callable only once (since the \ewpt{} is an affine resource itself).
The predicate \(W\) appears twice in the definition of the protocol.
Once in the precondition of \ocamlin{waker} and then in the postcondition of the whole protocol.
It signifies the resources that are transmitted from the party that calls the \ocamlin{waker} function to the fiber that performed the effect.
By appropriately instantiating \(W\), we can enforce that some condition holds before the fiber can be signalled to continue execution, and we get to assume the resources \(W\; v\) for the rest of the execution.

\subsubsection{Logical State}
\label{sec:sched-spec-state}

The most basic ghost state we define is a variation of a \emph{one-shot}, which we use in several places to track whether a reference \(l\) holding an optional value has been assigned to.
Its rules are described in figure~\ref{fig:promise-state-rules}.
Initially, we create two copies of \gspwait{\gamma}, which expresses that the reference holds a \(None\) value.
One copy can be placed into an invariant that either holds an \gspwait{\gamma} or an \(\gspdone{\gamma}~v\) along with the points-to connective of the reference \(l\).
Using the second copy, we can then differentiate the two cases of the invariant because the \gspwait{\gamma} and \gspdone{\gamma} resources cannot exist at the same time.
When assigning a value \(v\) to the reference, both copies are combined and converted to a persistent \(\gspdone{\gamma}~v\).
If the value does not matter we just write \gspdone{\gamma}.

Other pieces of ghost state are \gsPInvIn{}, \gsIsPr{}, \gsMainResIn{}, and \gsReady{1} described in figure~\ref{fig:logical-state-simpl}.

\gsPInvIn{} tracks additional resources for all existing promises by using an authoritative map which contains for each promise:
a location \(p\) holding its current program value,
a ghost name \(\gamma\) that is used for the \gspwait{\gamma} and \gspdone{\gamma} resources,
and a predicate \(Φ\) that describes the value the promise will eventually hold.
Additionally, for each promise in the map we own resources as part of \gsPInvIn{} that depend on the current state of the promise.
As long as the promise is not fulfilled we know that \(bcst\) is a broadcast instance, and we own one copy of \gspwait{\gamma} and a \gssignal{}.
The \gssignal{} is used to call the \ocamlin{Broadcast.signal_all} function which must only be called once.
When the promise is fulfilled, we instead own an \gspdone{\gamma}, and we know that the final value satisfies the given postcondition \(Φ\).

We define \gsPInv{} as an invariant that contains the promise map so that we can globally share it.
\gsIsPr{} represents the knowledge that a certain promise is contained in the map of \gsPInvIn{} and can be used to temporarily access the resources of this promise.
The \(\gamma_p\) ghost name is globally unique to identify the global map of promises.

We take a similar approach for the result of the main fiber but this resource exists for each scheduler instead of being globally unique.
\(\gsMainResIn{}~\gamma~l_{res}~\Phi\) tracks the state of the location \(l_{res}\) (\ocamlin{result} in figure~\ref{fig:sched-impl-run}).
The location either contains \(None\) or a value that satisfies the postcondition of the main fiber \(\Phi\).

The \gsReady{1} predicate is used as the invariant for each scheduler's \ocamlin{run_queue}.
It is parameterized by the ghost name \(\gamma\) of the scheduler's \gsMainResIn{} resource.
\(\gsReady{1}~\gamma\) expresses that all fibers are safe to execute and will only return when the result of the main fiber has been assigned (hence the \gspdone{\gamma}).
This formulation is due to the continuation passing style construction of the scheduler, which invokes a continuation at the end of the \ocamlin{execute} function, so the function only returns when all fibers have finished.

\begin{figure}
  \begin{mathpar}
    \textsc{OneShotV} \triangleq \textsc{Frac} \csumm \textsc{Ag}(val)
    \and
    %
    \mathrm{Persistent}(\gspdone{\gamma}~v)\\
    %
    \gspwait{\gamma} \triangleq \ownGhost{\gamma}{\frac{1}{2}}
    \and
    \gspdone{\gamma}~v \triangleq \ownGhost{\gamma}{\aginj{}(v)} \\
    \inferrule[OS-Create]
    { }
    { \upd \exists \gamma.\; \gspwait{\gamma} \ast \gspwait{\gamma} }
    % TODO this update modality looks horrible
    % 
    \and
    % 
    \inferrule[OS-Combine]
    { \gspwait{\gamma} \ast \gspwait{\gamma} }
    { \always \gspdone{\gamma}~v }
    % 
    \and
    % 
    \inferrule[OS-Contra]
    { \gspwait{\gamma} \ast \gspdone{\gamma}~v }
    { \bot }
    %
  \end{mathpar}
  \caption{Rules for the one-shot assignment resource.}
  \label{fig:promise-state-rules}\label{spec:ps_contra}
\end{figure}

\begin{figure}[ht]
  \begin{align*}
    \gsPState{}\; p\; \gamma\; \Phi    & \triangleq\; \begin{aligned}[t]
                                                               & (\exists\, bcst.\; p \mapsto Waiting\; bcst \ast \gspwait{\gamma} \ast \gsIsBcst{}\; bcst \ast \gssignal{}) \\
                                                        \vee\; & (\exists\, v.\; p \mapsto Done\; v \ast \gspdone{\gamma} \ast \always \Phi\; v)
                                                      \end{aligned}               \\
    \gsPInvIn{}                        & \triangleq\; \exists M.\; \ownGhost{\gamma_p}{\authfull{}\; M} \ast \forall\, (p, \gamma) \mapsto \Phi \in M.\; \gsPState{}\; p\; \gamma\; \Phi \\
    \gsIsPr{}\; \gamma\; p\; \Phi      & \triangleq\; \ownGhost{\gamma_p}{\authfrag{}\; \left\{\left[(p, \gamma) \mapsto \Phi\right]\right\}}                                            \\
    \gsPInv{}                          & \triangleq\; \knowInv{\pN{}}{\gsPInvIn{}}                                                                                                       \\
    \gsMainResIn{}~\gamma~l_{res}~\Phi & \triangleq\; \begin{aligned}[t]
                                                               & (l_{res} \mapsto None \ast \gspwait{\gamma})                                      \\
                                                        \vee\; & (\exists\, v.\; l_{res} \mapsto Some~v \ast \gspdone{\gamma} \ast \always \Phi~v)
                                                      \end{aligned}                                         \\
    \gsMainRes{}~\gamma~l_{res}~\Phi   & \triangleq\; \knowInv{\rN{}}{\gsMainRes{}~\gamma~l_{res}~\Phi}                                                                                  \\
    \gsReady{1}~\gamma~f               & \triangleq\;                   \ewp{f\; ()}{\bot}{\gspdone{\gamma}}
  \end{align*}
  \caption{Logical state definitions for the verification of our Eio model.}
  \label{fig:logical-state-simpl}\label{spec:pinv}\label{spec:is_promise}\label{spec:pstate}
\end{figure}

In the next sections we discuss the specifications we proved for the three functions.
We show a detailed proof of the specification only for \ocamlin{Promise.await} because it is the most involved.

\subsubsection{\ocamlin{Scheduler.run}}
\label{sec:sched-spec-run}

The interesting part about the scheduler specification \textsc{Spec-Run} is that it proves \textbf{effect safety} of the fiber runtime, i.e. no matter what a fiber does it will not crash the scheduler due to an unhandled effect.
This is expressed by allowing the fiber \(main\) to perform effects according to the \proto{1} protocol, but running the scheduler on the main fiber (\(run~main\)) obeys the empty protocol, so no effects escape.
Of course, the \ewpt{} itself also implies \textbf{safety} of running both the main fiber and the scheduler.

\[
  \inferrule[Spec-Run]
  {\ewp{main\; ()}{\proto{1}}{v.\; \always \Phi\; v}}
  {\ewp{run\; main}{\bot}{v.\; \always \Phi\; v}}
\]

Regarding the safety of matching on the \ocamlin{result} reference: Because the \ocamlin{execute} function only returns when the main fiber has finished (so it has also assigned a value to \ocamlin{result}),
we show that the postcondition of \ocamlin{execute} includes \(\gspdone{\gamma}\), which allows us to refute the error branch of the final match expression since according to \(\gsMainResIn{}\) the reference is assigned some value.

\subsubsection{\ocamlin{Fiber.fork_promise}}
\label{sec:sched-spec-fork}

The specification \textsc{Spec-ForkPromise} expresses that we receive from \(fork\_promise\) a promise \(p\) that will eventually hold a value satisfying \(\Phi\).
It has two preconditions, for one we must give it an arbitrary expression \(f\) representing the new fiber.
When called on unit, \(f\) obeys the \proto{1} protocol and returns some value \(v\) satisfying \(\Phi\).
Also, \(fork\_promise\) needs the \gsPInv{} invariant to interact with the global collection of promises, because it creates a new promise and fulfills it after \(f\) has finished execution.

\[
  \inferrule[Spec-ForkPromise]
  {\gsPInv{} \ast \ewp{f\;()}{\proto{1}}{v.\; \always \Phi\; v}}
  {\ewp{fork\_promise\; f}{\proto{1}}{p.\; \exists \gamma.\; \gsIsPr{}\; \gamma\; p\; \Phi}}
\]

\subsubsection{\ocamlin{Promise.await}}
\label{sec:sched-spec-await}

The specification \textsc{Spec-Await} is the direct counterpart to \textsc{Spec-ForkPromise}.
It shows that \(await\) consumes a promise \(p\) and eventually returns its value \(v\) satisfying the predicate \(\Phi\).
The precondition \gsPInv{} is again necessary to interact with the global collection of promises and \gsIsPr{} is used to identify the promise \(p\) in that collection.

If \(p\) is still unfulfilled the first time \(await\) checks the promise state, it calls \(make\_register\) to create a \ocamlin{register} function which it passes to the \esuspend{} effect.
As the \textsc{Spec-MakeRegister} specification shows, \(make\_register\) returns a suitable function that satisfies the \gsIsReg{1} predicate, instantiating \(W\) with \((\lambda\; v.\; \ulcorner v = () \urcorner \ast \gspdone{\gamma})\) so that we obtain an \gspdone{\gamma} resource when the effect returns.
This then allows us to refute the error case in the final match.

\begin{mathpar}
  \inferrule[Spec-MakeRegister]
  { \gsPInv{} \ast \gsIsPr{}\; \gamma\; p\; \Phi \ast \gsIsBcst{}\; bcst }
  { \ewp{make\_register\; p\; bcst}{\bot}{reg.\; \gsIsReg{1}\; reg\; (\lambda\; v.\; \ulcorner v = () \urcorner \ast \gspdone{\gamma})}}\label{spec:make_register}
  % 
  \and
  % 
  \inferrule[Spec-Await]
  { \gsPInv{} \ast \gsIsPr{}\; \gamma\; p\; \Phi }
  { \ewp{await\; p}{\proto{1}}{v.\; \always \Phi\; v}}\label{spec:await}
\end{mathpar}

In figures~\ref{fig:sched-spec-makeregister-proof} and~\ref{fig:sched-spec-await-proof} we give Hoare-style proof annotations for the two functions \(make\_register\) and \(await\).
The proof of \textsc{Spec-MakeRegister} uses the specifications of some broadcast functions.
We briefly explain these specifications and their logical state definitions now and expand upon them in section~\ref{sec:broadcast-spec}.

\begin{align*}
  \gsIsCb{}\; cb\; R                            & \triangleq R \wand \ewp{cb\; ()}{\bot}{\top}                                                                    \\
  \emph{isBroadcastRegisterResult}\; r\; cb\; R & \triangleq \begin{aligned}[t]
                                                                      & (\ulcorner r = Invoked \urcorner)                                                           \\
                                                               \vee\; & (\ulcorner r = Registered\; h \urcorner \ast \emph{isBroadcastRegisterHandle}\; h\; cb\; R)
                                                             \end{aligned} \\
  \emph{isBroadcastRegisterHandle}              & : Val \to Val \to iProp \to iProp
\end{align*}

\begin{mathpar}
  \inferrule[Spec-BroadcastRegister]
  { \gsIsBcst{}\; bcst \ast \gsIsCb{}\; callback\; R }
  { \ewp{register\; bcst\; callback}{\bot}{r.\; \emph{isBroadcastRegisterResult}\; r\; callback\; R}}\label{spec:bcst_register}
  % 
  \and
  % 
  \inferrule[Spec-BroadcastTryCancel]
  { \emph{isBroadcastRegisterHandle}\; h\; cb\; R }
  { \ewp{try\_unregister\; h}{\bot}{b.\; if\; b\; then\; \gsIsCb{}\; cb\; R\; else\; \top}}\label{spec:bcst_cancel}
\end{mathpar}

The function \ocamlin{Broadcast.register} takes a callback \(cb\) that satisfies the \gsIsCb{} predicate to register it in the broadcast data structure.
This predicate is structurally similar to \gsIsWaker{} and, in fact, in the proof of \textsc{Spec-MakeRegister} we instantiate the precondition \(R\) with \(\gspdone{\gamma}\) and pass as the callback a \ocamlin{waker} function, which has the precondition \((\lambda\; v.\; \ulcorner v = () \urcorner \ast \gspdone{\gamma})\) as described above.
The result of \ocamlin{Broadcast.register} is either a value \(Invoked\), which expresses that it called the callback directly, or a register handle, which can be used to call \ocamlin{Broadcast.try_unregister}.

\ocamlin{Broadcast.try_unregister} attempts to cancel a previous registration identified by the given \(handle\).
If the cancellation is successful, we receive a \(\gsIsCb{}\) resource which shows that we can safely call the callback again.

\paragraph*{Hoare-Style Proofs for \textsc{Spec-MakeRegister} and \textsc{Spec-Await}}

In the proof below an opened invariant \(Inv\) is represented as \(\cancel{Inv}\) and resources that are not needed for the rest of the proof are dropped implicitly.

The proof of \textsc{Spec-MakeRegister} is straightforward and follows from the specifications of \ocamlin{Broadcast.register} and \ocamlin{Broadcast.try_unregister}.
For \textsc{Spec-Await}, the crux is that we define \textsc{Spec-MakeRegister} so that it returns a \(register\) function which satisfies \(\gsIsReg{1}\; register\; (\lambda\; v.\; \ulcorner v = () \urcorner \ast \gspdone{\gamma})\).
Then, we get access to the \(\gspdone{\gamma}\) resource when the \esuspend{} effect returns, and we can refute the case of the promise still being unfulfilled when checking the state of promise again for the last time.

\begin{figure}[H]
  \[
    \inferrule[Spec-MakeRegister]
    { \gsPInv{} \ast \gsIsPr{}\; \gamma\; p\; \Phi \ast \gsIsBcst{}\; bcst }
    { \ewp{make\_register\; p\; bcst}{\bot}{reg.\; \gsIsReg{1}\; reg\; (\lambda\; v.\; \ulcorner v = () \urcorner \ast \gspdone{\gamma})}}
  \]
  \hspace*{-5em}{\setlength{\extrarowheight}{3pt}
    \begin{tabular}{@{}ll@{}}
      \ocamlreal{let make_register (p: 'a t) (bcst: Broadcast.t)}                                             &                                                                                                               \\
      \myquad[2] \ocamlreal{: (unit waker -> unit) =}                                                         &                                                                                                               \\
      \(\left\{ \gsPInv{} \ast \gsIsPr{}\; \gamma\; p\; \Phi \ast \gsIsBcst{}\; bcst \right\}\)               &                                                                                                               \\
      \myquad[1] \ocamlreal{  fun (waker: unit waker) ->}                                                     & [ intro waker that satisfies \hyperref[spec:is_waker]{\gsIsWaker{}} ]                                         \\
      \(\left\{ \makecell{\gsPInv{} \ast \gsIsPr{}\; \gamma\; p\; \Phi \ast \gsIsBcst{}\; bcst\; \ast                                                                                                                           \\ (\gspdone{\gamma} \wand \ewp{waker\; ()}{\bot}{\top}) } \right\} \)&\\
      \myquad[2] \ocamlreal{  let regres = Broadcast.register bcst waker in}                                  & [ apply \hyperref[spec:bcst_register]{\textsc{Spec-BroadcastRegister}} with \(R \coloneq \gspdone{\gamma}\) ] \\
      \(\left\{ \makecell{\gsPInv{} \ast \gsIsPr{}\; \gamma\; p\; \Phi \ast \gsIsBcst{}\; bcst\; \ast                                                                                                                           \\ \emph{isBroadcastRegisterResult}\; regres } \right\}\) &\\
      \myquad[2] \ocamlreal{  match regres with}                                                              & [ case analysis on \(regres\) ]                                                                               \\[3pt]
      \hline                                                                                                                                                                                                                  \\[-15pt]
      1. \(\left\{  regres = None \right\}\)                                                                  &                                                                                                               \\
      \myquad[2] \ocamlreal{ | None -> () }                                                                   & [ goal is {\color{red}trivial} ]                                                                              \\
      \hphantom{1..} \(\left\{ {\color{red}\top} \right\}\)                                                   &                                                                                                               \\[3pt]
      \hline                                                                                                                                                                                                                  \\[-12pt]
      2. \(\left\{ \makecell{ \gsPInv{} \ast \gsIsPr{}\; \gamma\; p\; \Phi \ast \gsIsBcst{}\; bcst\; \ast                                                                                                                     \\ regres = Some\; handle \ast \emph{isBroadcastRegisterHandle}\; handle } \right\}\) & \\
      \myquad[2] \ocamlreal{ | Some handle -> }                                                               & [ open \hyperref[spec:pinv]{\gsPInv{}}, lookup \(p\) using \hyperref[spec:is_promise]{\gsIsPr{}} ]            \\
      \hphantom{2..} \(\left\{ \makecell{ \cancel{\gsPInv{}} \ast \gsIsBcst{}\; bcst\; \ast                                                                                                                                   \\ \emph{isBroadcastRegisterHandle}\; handle \ast \gsPState{}\; p\; \gamma\; \Phi } \right\}\) &\\
      \myquad[3] \ocamlreal{ match Atomic.get p with }                                                        & [ case analysis on \hyperref[spec:pstate]{\gsPState{}} ]                                                      \\[3pt]
      \hline                                                                                                                                                                                                                  \\[-12pt]
      2.1. \(\left\{ \makecell{ \cancel{\gsPInv{}} \ast \gsIsBcst{}\; bcst\; \ast                                                                                                                                             \\ \emph{isBroadcastRegisterHandle}\; handle\; \ast \\ p \mapsto Done\; result \ast \gspdone{\gamma} } \right\}\)  &\\
      \myquad[3] \ocamlreal{| Done result -> }                                                                & [ close \hyperref[spec:pinv]{\gsPInv{}} ]                                                                     \\
      \hphantom{2.1..} \(\left\{ \makecell{ \gsIsBcst{}\; bcst \ast \emph{isBroadcastRegisterHandle}\; handle\; \ast                                                                                                            \\ \gspdone{\gamma} } \right\}\) &\\
      \myquad[4] \ocamlreal{ if Broadcast.try_unregister handle }                                             & [ apply \hyperref[spec:bcst_cancel]{\textsc{Spec-BroadcastTryCancel}}, case analysis on return value  ]       \\[3pt]
      \hline                                                                                                                                                                                                                  \\[-15pt]
      2.1.1. \(\left\{ \gspdone{\gamma} \ast (\gspdone{\gamma} \wand \ewp{waker\; ()}{\bot}{\top}) \right\}\) & [ specialize assumption ]                                                                                     \\
      \hphantom{2.1.1..} \(\left\{ {\color{red}\ewp{waker\; ()}{\bot}{\top}} \right\}\)                       &                                                                                                               \\
      \myquad[4] \ocamlreal{ then waker () }                                                                  & [ by {\color{red}apply} \(\ewp{waker\; ()}{\bot}{\top}\) ]                                                    \\
      \hphantom{2.1.1..} \(\left\{ {\color{red}\top} \right\}\)                                               &                                                                                                               \\[3pt]
      \hline                                                                                                                                                                                                                  \\[-15pt]
      2.1.2. \(\left\{ \top \right\}\)                                                                        &                                                                                                               \\
      \myquad[4] \ocamlreal{ else () }                                                                        & [ goal is {\color{red}trivial} ]                                                                              \\
      \hphantom{2.1.2..} \(\left\{ {\color{red}\top} \right\}\)                                               &                                                                                                               \\[3pt]
      \hline                                                                                                                                                                                                                  \\[-15pt]
      2.2. \(\left\{ \cancel{\gsPInv{}} \ast p \mapsto Waiting\; \_  \right\}\)                               &                                                                                                               \\
      \myquad[3] \ocamlreal{| Waiting _ -> () }                                                               & [ close \hyperref[spec:pinv]{\gsPInv{}}, goal is {\color{red}trivial} ]                                       \\
      \hphantom{2.2..} \(\left\{ {\color{red}\top} \right\}\)                                                 &
    \end{tabular}}
  \caption{Annotated proof of \hyperref[spec:make_register]{\textsc{Spec-MakeRegister}}.}
  \label{fig:sched-spec-makeregister-proof}
\end{figure}

\begin{figure}[H]
  \[
    \inferrule[Spec-Await]
    { \gsPInv{} \ast \gsIsPr{}\; \gamma\; p\; \Phi }
    { \ewp{await\; p}{\proto{1}}{v.\; \always \Phi\; v}}
  \]
  \hspace*{-3em}{\setlength{\extrarowheight}{3pt}
    \begin{tabular}{@{}ll@{}}
      \ocamlreal{ let await (p: 'a t) : 'a = }                                                                                       &                                                                                                                                    \\
      \( \left\{ \gsPInv{} \ast \gsIsPr{}\; \gamma\; p\; \Phi \right\} \)                                                            & [ open \hyperref[spec:pinv]{\gsPInv{}}, lookup \(p\) using \hyperref[spec:is_promise]{\gsIsPr{}} ]                                 \\
      \( \left\{ \makecell{ \cancel{\gsPInv{}} \ast \gsIsPr{}\; \gamma\; p\; \Phi \ast \gsPState{}\; p\; \gamma\; \Phi } \right\} \) &                                                                                                                                    \\
      \myquad[1] \ocamlreal{match Atomic.get p with}                                                                                 & [ case analysis on \hyperref[spec:pstate]{\gsPState{}} ]                                                                           \\[3pt]
      \hline                                                                                                                                                                                                                                                              \\[-12pt]
      1. \( \left\{ \makecell{ \cancel{\gsPInv{}} \ast                                                                                                                                                                                                                    \\ p \mapsto Done\; result \ast \always (\Phi\; result) } \right\} \)                                                       &                             \\
      \myquad[1] \ocamlreal{| Done result -> }                                                                                       & [ close \hyperref[spec:pinv]{\gsPInv{}} ]                                                                                          \\
      \hphantom{1..} \( \left\{ \gsPInv{} \ast {\color{red}\always (\Phi\; result)} \right\} \)                                      &                                                                                                                                    \\
      \myquad[2] \ocamlreal{result}                                                                                                  & [ by {\color{red}assumption} ]                                                                                                     \\
      \hphantom{1..} \( \left\{ {\color{red}\always (\Phi\; result)} \right\} \)                                                     &                                                                                                                                    \\[3pt]
      \hline                                                                                                                                                                                                                                                              \\[-12pt]
      2. \( \left\{ \makecell{ \cancel{\gsPInv{}} \ast \gsIsPr{}\; \gamma\; p\; \Phi\; \ast                                                                                                                                                                                 \\ p \mapsto Waiting\; bcst \ast \gsIsBcst{}\; bcst } \right\} \) &                                                  \\
      \myquad[1] \ocamlreal{| Waiting bcst ->}                                                                                       & [ close \hyperref[spec:pinv]{\gsPInv{}} ]                                                                                          \\
      \hphantom{2..} \( \left\{ \makecell{ \gsPInv{} \ast \gsIsPr{}\; \gamma\; p\; \Phi\; \ast                                                                                                                                                                              \\ \gsIsBcst{}\; bcst } \right\} \) &                                                  \\
      \myquad[2] \ocamlreal{let register = make_register p bcst}                                                                     & [ apply \hyperref[spec:make_register]{\textsc{Spec-MakeRegister}} ]                                                                \\
      \hphantom{2..} \( \left\{ \makecell{ \gsPInv{} \ast \gsIsPr{}\; \gamma\; p\; \Phi\; \ast                                                                                                                                                                              \\ \gsIsReg{1}\; register\; (\lambda v.\; \ulcorner v = () \urcorner \ast \gspdone{\gamma}) } \right\} \) &                                                  \\
      \myquad[2] \ocamlreal{perform (Suspend register);}                                                                             & [ protocol of \hyperref[spec:suspend]{\esuspend{}} with \((W := \lambda v.\; \ulcorner v = () \urcorner \ast \gspdone{\gamma})\) ] \\
      \hphantom{2..} \( \left\{ \makecell{ \gsPInv{} \ast \gsIsPr{}\; \gamma\; p\; \Phi\; \ast                                                                                                                                                                              \\ \gspdone{\gamma} } \right\} \) & [ open \hyperref[spec:pinv]{\gsPInv{}}, lookup \(p\) using \hyperref[spec:is_promise]{\gsIsPr{}} ] \\
      \hphantom{2..} \( \left\{ \makecell{ \cancel{\gsPInv{}} \ast \gspdone{\gamma}\; \ast                                                                                                                                                                                       \\ \gsPState{}\; p\; \gamma\; \Phi } \right\} \) & \\
      \myquad[2] \ocamlreal{match Atomic.get p with}                                                                                 & [ case analysis on \hyperref[spec:pstate]{\gsPState{}} ]                                                                           \\[3pt]
      \hline                                                                                                                                                                                                                                                              \\[-12pt]
      2.1.  \( \left\{ \makecell{ \cancel{\gsPInv{}}\; \ast                                                                                                                                                                                                               \\ p \mapsto Done\; result \ast \always (\Phi\; result) } \right\} \) &                                                  \\
      \myquad[2] \ocamlreal{| Done result -> }                                                                                       & [ close \hyperref[spec:pinv]{\gsPInv{}} ]                                                                                          \\
      \hphantom{2.1..}  \( \left\{ \gsPInv{} \ast {\color{red}\always (\Phi\; result)}  \right\} \)                                  &                                                                                                                                    \\
      \myquad[3] \ocamlreal{result}                                                                                                  & [ by {\color{red}assumption} ]                                                                                                     \\
      \hphantom{2.1..}  \( \left\{ {\color{red}\always (\Phi\; result)} \right\} \)                                                  &                                                                                                                                    \\[3pt]
      \hline                                                                                                                                                                                                                                                              \\[-12pt]
      2.2.  \( \left\{ \makecell{ \cancel{\gsPInv{}} \ast \gspdone{\gamma}\; \ast                                                                                                                                                                                         \\ p \mapsto Waiting\; bcst \ast \gspwait{\gamma} } \right\} \) &                                                  \\
      \myquad[2] \ocamlreal{| Waiting _ -> }                                                                                         & [ specialize \hyperref[spec:ps_contra]{\textsc{PS-Contra}} ]                                                                       \\
      \hphantom{2.2..}  \( \left\{ \cancel{\gsPInv{}} \ast {\color{red}\bot} \right\} \)                                             &                                                                                                                                    \\
      \myquad[3] \ocamlreal{error "impossible"}                                                                                      & [ by {\color{red}contradiction} ]                                                                                                  \\
      \hphantom{2.2..}  \( \left\{ {\color{red}\bot} \right\} \)                                                                     &
    \end{tabular}}
  \caption{Annotated proof of \hyperref[spec:await]{\textsc{Spec-Await}}.}
  \label{fig:sched-spec-await-proof}
\end{figure}

% The proof of the \ocamlin{Promise.await} specification proceeds as follows:
% \begin{itemize}
%   \item For the first match on the promise state we don't have any resources to constrain the possible results.
%   \item If the promise is already fulfilled we can take the \ocamlin{Φ v} and return that.
%   \item If it is not fulfilled, then we get access to a CQS instance and can make the \ocamlin{register} function using the \emph{IsPromise} and \ocamlin{is_cqs} resources.
%   \item Using the \ewpt{} for the \ocamlin{register} function we can invoke the \esuspend{} effect and set \ocamlin{W _ := promise_done γ}.
%   \item As a result we now have the \gspdone{\gamma} resource and when we match on the promise again, the unfulfilled case can be ruled out.
%   \item So now we can take the \ocamlin{Φ v} and return it.
% \end{itemize}

% The proof of the \ocamlin{make_register} specification follows directly from the specifications of the CQS functions, which are explained in further detail in the next chapter.

% <!-- We recall that when awaiting a promise, a fiber first checks if the promise is already fulfilled by atomically loading its state.
% If it is not fulfilled, the fiber then performs a \esuspend{} effect and starts a suspend operation, providing the \ocamlin{waker} of the \esuspend{} effect as the handle.
% The suspend operation might fail because the promise could have been fulfilled concurrently.
% Since the promise could have been fulfilled in the meantime, the fiber must then again atomically load the state of the promise.

% - If it has not been fulfilled the fiber does not need to do anything because it will eventually be woken up by a resume all operation invoking the \ocamlin{waker}.
% - But if the promise has been fulfilled the fiber must attempt to cancel the suspend operation.
%   That is because in this situation the suspend operation races with a concurrent resume all operation, which might already have invoked all \ocamlin{waker}s \textbf{before} this fiber was able to save its \ocamlin{waker} in the broadcast.
%   In this case the \ocamlin{waker} would be lost and the fiber never resumes execution.
%   If the \ocamlin{waker} has not been invoked yet (either because resume all has not arrived at this waker or it arrived before the waker was saved in the broadcast) the cancellation attempt succeeds and the fiber invokes its own \ocamlin{waker}.
%   Otherwise we know that the \ocamlin{waker} has already been invoked, so the fiber does not need to do anything.

% This complicated interplay between two fibers is due to CQS being lock-free but it ensures that fibers only resume execution when the promise is fulfilled and that all \ocamlin{waker}s will be eventually called. -->

% <!-- \ocamlin{}`
% Aside: All wakers are eventually called.
% This statement is purely based on a reading of the code. It might be possible to formally prove this with an approach
% like Iron~\cite{bizjak2019iron} or Transfinite Iris~\cite{spies2021transfinite} because it is a liveness property.
% But for the Iron approach it is unclear to us how to formulate the linearity property.
% \ocamlin{}` -->

% \subsubsection{Comparison of Logical State}
% \label{sec:sched-spec-state-comparison}

% \begin{figure}[ht]
%   \begin{align*}
%     \gsReady{1}\; q\; \phi\; k \triangleq \forall v.\; & \always \phi(v) \wand \later PromiseInv\; q \wand \later \gsIsQueue{}\; q\; (Ready\; q\; (\lambda w.\; w = ())) \wand \\
%                                                       & \ewp{k\; v}{\bot}{\top}
%   \end{align*}
% \end{figure}

% Since our logical state definitions are based on a case study of de Vilhena, we give a comparison of what had to be changed when adapting it to our model of Eio.
% First, in the original development the \gsReady{1} predicate fulfills two roles.
% \begin{enumerate}
%   \item It expresses that all continuations in the scheduler's run-queue are safe to execute.
%   \item It expresses that all continuations in a promise's waiting-queue are safe to execute.
% \end{enumerate}

% \gsPInv{} and \gsIsQueue{} were both necessary as preconditions because they are not persistent and need to be passed around explicitly.

% In our development \gsPInv{} could be dropped from the definition of Ready because it is now put into an Iris shareable invariant, and can be passed implicitly.
% Similarly, the \gsIsQueue{} precondition was dropped from the definition of \gsReady{1} because in Eio the run queue must be thread-safe, so the \gsIsQueue{} resource is persistent and can be passed to a fiber once when it is spawned.
% Therefore, our \gsReady{1} is neither recursive nor mutually recursive with \gsPInv{} anymore, which simplifies its usage in Iris.
% We note that the (mutual) recursion was only necessary because \gsPInv{} was used to track global state but was not put into an Iris shareable invariant, so it had to be passed around explicitly in many places.

% We also split up the two uses of \gsReady{1} and only use it under this name for the first role.
% In the case of a scheduler's run-queue, \(Φ\; v\) degenerates just to \(\ulcorner v = () \urcorner\), so we can drop both from the definition and use \(()\) directly.
% This is why our definition of \gsReady{1} only contains an \ewpt{} without preconditions.

% For the second use case of describing the continuations in a promise's waiting-queue we now have another specialized version of \gsReady{1}.
% As explained in the next section, a broadcast has an invariant \(W\; v \wand \ewp{callback\; ()}{\bot}{\top}\) for all stored \(callback\)s.
% This is just \gsReady{1} where \(W\; v\) replaces \(Φ\; v\), and it is the same \(W\) as in the definition of the \esuspend{} effect.
