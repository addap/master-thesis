% from https://tex.stackexchange.com/a/174628
\appendix
\addcontentsline{toc}{section}{Appendix}
\section*{Appendix}
\label{sec:appendix}

\renewcommand{\thesubsection}{\Alph{subsection}}

\subsection{Translation Table}
\label{sec:apdx-translation}

\begin{table}[ht]
  \begin{tabular}{l|l|l}
    Eio                          & Thesis                       & Mechanization          \\
    \hline
    \ocamlin{enqueue}            & \ocamlin{waker} function     & \ocamlin{waker}        \\
    \ocamlin{f}                  & \ocamlin{register} function  & \ocamlin{register}     \\
    \ocamlin{Fiber.fork_promise} & \ocamlin{Fiber.fork_promise} & \ocamlin{fork_promise} \\
    \ocamlin{Promise.await}      & \ocamlin{Promise.await}      & \ocamlin{await}        \\
    \ocamlin{Sched.run}          & \ocamlin{Scheduler.run}      & \ocamlin{run}          \\
  \end{tabular}
\end{table}

\subsection{Towards A Multi-Threaded Scheduler}
\label{sec:apdx-mt}

OCaml 5 added not only effect handlers but also the ability to use multiple threads of execution, which are called \textit{domains} (in the following we use the terms interchangeably).
Each domain in OCaml 5 corresponds to one system-level thread and the usual rules of multi-threaded execution apply, i.e. domains are preemtively scheduled and can share memory.
Eio defines an operation to make use of multi-threading by forking off a new thread and running a separate scheduler in it.
So while each Eio scheduler is only responsible for fibers in a single thread, fibers can await and communicate with fibers running in other threads.

In order for a fiber to be able to await fibers in another thread, the \ocamlin{wakers_queue} [note it will be in the Simple Scheduler section] from above is actually a thread-safe queue based on something called CQS, which we will discuss in detail in a later section.

Heaplang supports reasoning about multi-threaded programs by implementing fork and join operations for threads and defining atomic steps in the operational semantics, which enables the use of Iris \textit{invariants}.
In contrast, \hazel{} did not define any multi-threaded operational semantics but it contained most of the building blocks for using invariants.
In the following we explain how we added a multi-threaded operational semantics and enabled the use of invariants.

% \ocamlin{}`
% Aside: Memory Model in OCaml 5
% In the OCaml 5 memory model, \textit{atomic variables} are needed in order to access shared memory without introducing data races.
% Instead of modelling atomic variables in \hazel{}, we continue to use normal references because the multi-threaded operational semantics by definition defines all memory operations to be sequentially consistent. This seems to be the standard approach and is done the same way in Heaplang.
% \ocamlin{}`

\subsubsection*{Adding Invariants to \hazel{}}

Invariants in Iris are used to share resources between threads.
They encapsulate a resource to be shared and can be opened for a single atomic step of execution.
During this step the resource can be taken out of the invariant and used in the proof but at the end of the step the invariant must be restored.

\hazel{} did already have the basic elements necessary to support using invariants.
It defined a ghost cell to hold invariants and proved an invariant access lemma which allows opening an invariant if the current expression is atomic.
In order to use invariant we only had to provide proofs for which evaluation steps are atomic.
We provided proofs for all primitive evaluation steps.
The proofs are the same for all steps so we just explain the one for \ocamlin{Load}.

\begin{minted}{coq}
Lemma ectx_language_atomic a e :
head_atomic a e → sub_exprs_are_values e → Atomic a e.

Instance load_atomic v : Atomic StronglyAtomic (Load (Val v)).
Instance store_atomic v1 v2 : Atomic StronglyAtomic (Store (Val v1) (Val v2)).
...
\end{minted}

An expression is atomic if it takes one step to a value, and if all subexpressions are already values.
The first condition follows by definition of the step relation and the second follows by case analysis of the expression.

Since performing an effect starts a chain of evaluation steps to capture the current continuation, it is not atomic.
For the same reason an effect handler and invoking a continuation are not atomic except in degenerate cases.
Therefore, invariants and effects do not interact in any interesting way.
\todo{How we add support for the iInv tactic to use invariants more easily.}

\subsubsection*{Adding Multi-Threading to \hazel{}}

To allow reasoning in \hazel{} about multi-threaded programs we need a multi-threaded operational semantics as well as specifications for the new primitive operations \efork{}, \ocamlin{Cmpxcgh} and \ocamlin{FAA}.

The language interface of Iris provides a multi-threaded operational semantics that is based on a thread-pool.
The thread-pool is a list of expressions that represents threads running in parallel.
At each step, one expressions is picked out of the pool at random and executed for one thread-local step.
Each thread-local step additionally returns a list of forked-off threads, which are then added to the pool.
This is only relevant for the \efork{} operation as all other operations naturally don't fork off threads.

% (e, \sigma) ->_t (e', \sigma', es')
% ------------------------------------------------------------
% (es_1 ++ e ++ es_2, \sigma) ->_mt (es_1 ++ e' ++ es_2 + es', \sigma')

Heaplang implements multi-threading like this and for \hazel{} we do the same thing.
We adapt \hazel{}'s thread-local operational semantics to include \efork{}, \ocamlin{Cmpxchg} and \ocamlin{FAA} operations and to track forked-off threads and get a multi-threaded operational semantics "for free" from Iris' language interface.

Additionally, we need to prove specifications for these three operations.
\ocamlin{Cmpxchg} and \ocamlin{FAA} are standard so we will not discuss them here.
The only interesting design decision in the case of \hazel{} is how effects and \efork{} interact.
This decision is guided by the fact that in OCaml 5 effects never cross thread-boundaries.
An unhandled effect just terminates the current thread.
As such we must impose the empty protocol on the argument of \efork{}.

% EWP e <| \bot |> { \top }
% -------------------------------------
% EWP (Fork e) <| \Phi |> { x, x = () }

Using these primitive operations we can then build the standard \ocamlin{CAS}, \ocamlin{Spawn}, and \ocamlin{Join} operations on top and prove their specifications.
For \ocamlin{Spawn} \& \ocamlin{Join} we already need invariants as the point-to assertion for the done flag must be shared between the two threads.

% Lemma spawn_spec (Q : val → iProp Σ) (f : val) :
% EWP (f #()) <| \bot |> {{ Q }} -∗ EWP (spawn f) {{ v, ∃ (l: loc), ⌜ v = #l ⌝ ∗ join_handle l Q }}.

% Lemma join_spec (Q : val → iProp Σ) l :
% join_handle l Q -∗ EWP join #l {{ v, Q v }}.

% Definition spawn_inv (γ : gname) (l : loc) (Q : val → iProp Σ) : iProp Σ :=
% ∃ lv, l ↦ lv ∗ (⌜lv = NONEV⌝ ∨
% ∃ w, ⌜lv = SOMEV w⌝ ∗ (Q w ∨ own γ (Excl ()))).

% Definition join_handle (l : loc) (Q : val → iProp Σ) : iProp Σ :=
% ∃ γ, own γ (Excl ()) ∗ inv N (spawn_inv γ l Q).

Note that for \ocamlin{Spawn} we must also impose the empty protocol on \ocamlin{f} as this expression will be forked-off.

This allows us to implement standard multi-threaded programs which also use effect handlers.
For example, we can prove the specification of the function below that is based on an analogous function in Eio which forks a thread and runs a new scheduler inside it.
Note that same as in Eio the function blocks until the thread has finished executing, so it should be called in separate fiber.

% Definition spawn_scheduler : val :=
% (λ: "f",
% let: "new_scheduler" := (λ: <>, run "f") in
% let: "c" := spawn "new_scheduler" in
% join "c")%V.

% Lemma spawn_scheduler_spec (Q : val -> iProp Σ) (f: val) :
% promiseInv -∗ EWP (f #()) <| Coop |> {{ _, True }} -∗
% EWP (spawn_scheduler f) {{ _, True }}.

The scheduler \ocamlin{run} and therefore also the \ocamlin{spawn_scheduler} function don't have interesting return values, so this part of the specification is uninteresting.
What is more interesting is that they encapsulate the possible effects the given function \ocamlin{f} performs.

\subsection{A Note on Cancellation}
\label{sec:apdx-cancellation}

\begin{itemize}
  \item That we tried to model cancellation but the feature is too permissive to give it a specification.
  \item There is still an interesting question of safety (fibers cannot be added to a cancelled Switch).
  \item But including switches \& cancellation in our model would entail too much work so we leave it for future work.
\end{itemize}
