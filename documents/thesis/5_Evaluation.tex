\section{Evaluation}
\label{sec:evaluation}
We evaluate our model of Eio on a simple example program that uses all features supported by our implementation.
The example program (shown in figure~\ref{fig:eval_code}) may look contrived since it does not do any "useful" computation, but the value of Eio as a library comes from composing computations -- not in what is computed concretely.

The program's \ocamlin{main_fiber} function forks off a new fiber \ocamlin{dispatch} (line~\ref{ln:eval_dispatch}) and communicates with it over a one-element channel represented by the thread-local variable.
The channel is initially empty (first argument to \ocamlin{Scheduler.run} in line~\ref{ln:eval_none}) and \ocamlin{dispatch} polls for data (line~\ref{ln:eval_wait}).
\ocamlin{main_fiber} sends one integer \ocamlin{data} (lines~\ref{ln:eval_receive},\ref{ln:eval_send}) to \ocamlin{dispatch} which will then run two copies of \ocamlin{(work data)} (line~\ref{ln:eval_thread}) in separate threads, and sum their results.
The \ocamlin{work} function simulates time passing using the \ocamlin{yield} function (line~\ref{ln:eval_work}) and returns its first argument.
Yield performs a \esuspend{} effect but calls the \ocamlin{waker} function immediately, which has the effect of placing the current fiber at the back of the scheduler's run queue to give other fibers a chance to run.

The example program therefore uses the basic functions for forking and awaiting the completion of fibers, multiple schedulers running in different threads, as well as thread-local variables to communicate between fibers within one thread.

\begin{figure}[ht]
    \begin{minted}[escapeinside=@@]{ocaml}
let yield () = @\label{ln:eval_yield}@
  perform (Suspend (fun waker -> waker ()))

let work x () = 
  yield (); @\label{ln:eval_work}@
  x

let rec wait_for_read tlv =
  match !tlv with
  | None -> yield (); wait_for_read tlv
  | Some data -> tlv <- None; data @\label{ln:eval_receive}@

let dispatch () =
  let tlv = get_context ().tlv in
  let data = wait_for_data tlv in @\label{ln:eval_wait}@
  let p1 = Fiber.fork_promise (Domain_manager.run (work data)) in @\label{ln:eval_thread}@
  let p2 = Fiber.fork_promise (Domain_manager.run (work data)) in
  let r1 = Promise.await p1 in
  let r2 = Promise.await p2 in
  r1 + r2

let main_fiber data () =
  let p = Fiber.fork_promise dispatch in @\label{ln:eval_dispatch}@
  let tlv = get_context ().tlv in
  tlv <- Some data @\label{ln:eval_send}@
  Promise.await p

let main () = 
  Scheduler.run None (main_fiber 21) @\label{ln:eval_none}@
\end{minted}
    \caption{Example program to use all implemented features.}
    \label{fig:eval_code}
\end{figure}

We first give the final specifications of the most important components of our model library in figure~\ref{fig:eval-total}.
The specifications contain both extensions we discussed in sections~\ref{sec:domain-manager} and~\ref{sec:thread-local-vars}.

\begin{figure}[ht]
    \begin{align*}
        \gsReady{}~l~\Omega~\gamma~q~k & \triangleq \begin{aligned}[t]
                                                         & \gsFiberResources{l}{\Omega} \wand                                                                                                           \\
                                                         & \later isQueueReader~q~(\gsReady{}~l~\Omega~\gamma~q) \wand                                                                                  \\
                                                         & \quad \ewp{k~()}{\bot}{\_.\; \gsFiberResources{l}{\Omega} \ast \later isQueueReader~q~(\gsReady{}~l~\Omega~\gamma~q) \ast \gspdone{\gamma} }
                                                    \end{aligned} \\
        \gsIsWaker{}~wkr~W             & \triangleq \forall v.   \;  W\; v \wand{} \ewp{wkr\ \ v}{\bot}{\top}                                                                                                \\
        \gsIsReg{}~reg~W~S             & \triangleq \forall wkr. \; \gsIsWaker{}\; wkr\; W \wand{} \ewp{reg\ \ wkr}{\bot}{\always S}                                                                         \\
        \proto{}~l~\Omega              & \triangleq \begin{aligned}[t]
                                                        Fork\;       & \#\; \begin{aligned}[t]
                                     & !\; e\; ((l, e))\; \begin{aligned}[t]
                                           & \big\{ \gsFiberResources{l}{\Omega} \ast                                                                              \\
                                           & \later\; (\gsFiberResources{l}{\Omega} \wand{} \ewp{e\ \ ()}{\proto{}~l~\Omega}{\gsFiberResources{l}{\Omega}}) \big\}
                                      \end{aligned} . \\
                                     & ?\; ()\; \{ \gsFiberResources{l}{\Omega} \}
                                \end{aligned}                                             \\
                                                        Suspend\;    & \#\; \begin{aligned}[t]
                                     & !\; reg\; W\; S\; (reg)\; \{\gsFiberResources{l}{\Omega} \ast \gsIsReg{}\; reg\; W\; S\}. \\
                                     & ?\; v\; (v)\; \{ \gsFiberResources{l}{\Omega} \ast W~v \ast S \}
                                \end{aligned} \\
                                                        GetContext\; & \#\; !\; ()\; \{\top\}.\; ?\; (l)\; \{ \top \}
                                                    \end{aligned}
    \end{align*}
    \begin{mathpar}
        \inferrule[Spec-Run]
        { \Omega~init \;\ast \\\\ \forall l_{tlv}.\; \gsFiberResources{l_{tlv}}{\Omega} \wand \ewp{main\ \ ()}{\proto{}~l_{tlv}~\Omega}{v.\; \always \Phi~v \ast \gsFiberResources{l_{tlv}}{\Omega}} }
        { \ewp{run\ \ init\ \ main}{\bot}{v.\; \always \Phi~v \ast \gsFiberResources{l_{tlv}}{\Omega}} }
        %
        \and
        %
        \inferrule[Spec-ForkPromise]
        { \gsPInv{} \ast \gsFiberResources{l_{tlv}}{\Omega} \;\ast \\\\ \gsFiberResources{l_{tlv}}{\Omega} \wand \ewp{f\ \ ()}{\proto{}~l_{tlv}~\Omega}{v.\; \always \Phi~v \ast \gsFiberResources{l_{tlv}}{\Omega}} }
        { \ewp{fork\_promise\ \ f}{\proto{}~l_{tlv}~\Omega}{p.\; \gsIsPr{}~p~\Phi \ast \gsFiberResources{l_{tlv}}{\Omega}} }
        %
        \and
        %
        \inferrule[Spec-Await]
        { \gsPInv{} \ast \gsFiberResources{l_{tlv}}{\Omega} \ast \gsIsPr{}~p~\Phi }
        { \ewp{await\ \ p}{\proto{}~l_{tlv}~\Omega}{p.\; \gsIsPr{}~p~\Phi \ast \gsFiberResources{l_{tlv}}{\Omega}} }
        %
        \and
        %
        \inferrule[Spec-NewScheduler]
        { \Omega'~init \;\ast \\\\ \forall l'_{tlv}.\; \gsFiberResources{l'_{tlv}}{\Omega} \wand \ewp{main\ \ ()}{\proto{}~l'_{tlv}~\Omega'}{v.\; \always \Phi~v \ast \gsFiberResources{l'_{tlv}}{\Omega'}} }
        { \ewp{new\_scheduler\ \ init\ \ main}{\proto{}~l~\Omega}{v.\; \always \Phi~v \ast \gsFiberResources{l}{\Omega}} }
    \end{mathpar}

    \caption{Specification of the public interface of the Eio library model.}
    \label{fig:eval-total}
\end{figure}

Using these specifications we proved the safety of the example program and its functional correctness by establishing specifications for each function as shown in figure~\ref{fig:eval-spec}.
We can see that there is some amount of boilerplate (colored in {\color{blue} blue}).
Each function that yields execution to another fiber by performing an effect -- either directly or indirectly through another function call -- needs \({\color{blue}\gsFiberResources{l_{tlv}}{\Omega}}\), which signifies the ownership over the thread-local variable.
Additionally, any fiber that wants to fork off another fiber using \ocamlin{Fiber.fork_promise} needs the persistent \({\color{blue}\gsPInv{}}\) resource to interact with the global collection of promises.
The predicate \(\Omega_{chan}~\gamma\) as part of \(\gsFiberResources{l_{tlv}}{(\Omega_{chan}~\gamma)}\) restricts the thread-local variable \(l_{tlv}\) to be a channel for a single message \(n\).

\begin{figure}[ht]
    \begin{align*}
        \Omega_{chan}~\gamma~v & \triangleq \begin{aligned}[t]
                                                       & \ulcorner v = None \urcorner                                                    \\[-5pt]
                                                \vee\; & \exists n.\; \ulcorner v = Some~n \urcorner \ast \ownGhost{\gamma}{\aginj{}(n)}
                                            \end{aligned} \\
    \end{align*}
    \begin{mathpar}
        \inferrule[Spec-Work]
        { {\color{blue}\gsFiberResources{l_{tlv}}{\Omega}} }
        { \ewp{work\ \ n\ \ ()}{\proto{}~l_{tlv}~\Omega}{v.\; \ulcorner v = n \urcorner \ast {\color{blue}\gsFiberResources{l_{tlv}}{\Omega}}} }
        %
        \and
        %
        \inferrule[Spec-WaitForData]
        { {\color{blue}\gsFiberResources{l_{tlv}}{(\Omega_{chan}~\gamma)}} }
        { \ewp{wait\_for\_data\ \ l}{\proto{}}{v.\; \exists n.\; \ulcorner v = n \urcorner \ast \ownGhost{\gamma}{\aginj{}(n)} \ast {\color{blue}\gsFiberResources{l_{tlv}}{(\Omega_{chan}~\gamma)}}} }
        %
        \and
        %
        \inferrule[Spec-Dispatch]
        { {\color{blue}\gsPInv{}} \ast {\color{blue}\gsFiberResources{l_{tlv}}{(\Omega_{chan}~\gamma)}} }
        { \ewp{dispatch\ \ ()}{\proto{}}{v.\; \exists n.\; \ulcorner v' = n + n \urcorner \ast \ownGhost{\gamma}{\aginj{}(n)} \ast {\color{blue}\gsFiberResources{l_{tlv}}{(\Omega_{chan}~\gamma)}}} }
        %
        \and
        %
        \inferrule[Spec-MainFiber]
        { {\color{blue}\gsPInv{}} \ast {\color{blue}\gsFiberResources{l_{tlv}}{(\Omega_{chan}~\gamma)}} \ast \ownGhost{\gamma}{\aginj{}(n)} }
        { \ewp{main\_fiber\  n\  ()}{\proto{}}{v.\; \ulcorner v = n + n \urcorner \ast {\color{blue}\gsFiberResources{l_{tlv}}{(\Omega_{chan}~\gamma)}}} }
        %
        \and
        % 
        \inferrule[Spec-Main]
        { }
        { \ewp{main\ \ ()}{\bot}{v.\; \ulcorner v = 42 \urcorner} }
    \end{mathpar}
    \caption{Specification of the example program.}
    \label{fig:eval-spec}
\end{figure}

While we proved the safety of this program (and of the core abstractions of the Eio library), the complete Eio library has more features that are still unexplored.
This includes cancellation of fibers, resource management using switches and several operating system primitives like timers, so we cannot make any statements about programs using these features.
Nevertheless, our model is an important step in the direction of proving the safety of Eio and programs that use it.
Iris along with its features like ghost state and shareable invariants to reason about multithreaded and stateful code were key in this development.

