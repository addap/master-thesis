\begin{abstract}
    % General topic & Specific topic
    In this thesis we work on the formal verification of the OCaml library \emph{Eio} which provides user-level concurrency using the new effect handlers feature of \ocf{}.
    % General Question of the Problem
    As part of formal verification, the goal of program verification is to show that a program obeys a specification and is safe to execute, meaning that its execution will not run into any undefined behavior or crash.
    % Previous Research
    Program verification for languages with mutable state is commonly done using separation logics.
    For reasoning about effect handlers there exists the ML-like language \hh{} and an associated program logic called \hazel{}, built on top of the Iris separation logic framework.

    % Goal of the Research
    We tackle the question of safety for the central elements of the Eio library, which includes \emph{spawning fibers} that are \emph{run by a scheduler} and can wait for the completion of other fibers by \emph{awaiting promises}.
    % Reason
    Therefore, our work serves as an extended case study on the usefulness of modelling and verifying programs with effect handlers in \hazel{}.
    % Methods
    The formal verification is carried out in the Hazel logic and our results are mechanized in the Coq proof assistant.

    % Result
    We were able to verify the safety of the central elements of the Eio library, and prove specifications for its public API and for the declared effects.
    We also extended the \hh{} language to include multithreading in order to adapt previous verification work on a data structure that Eio uses.
\end{abstract}