\section{Adding Support for Thread-Local Variables}
\label{sec:thread-local-vars}

% General information about the GetContext effect and that it's used to implement thread-local variables

Previously we have looked at a protocol with two effects which suffice to model fibers that can suspend and fork off new fibers.
But fibers in Eio can use an additional effect called \egetctx{} that we discuss in this section.
For each fiber the scheduler keeps track of context metadata, one part of which are \emph{thread-local variables}.
Thread-local variables are state that is shared between all fibers of one scheduler (hence thread-local) and a fiber gets access to them via the \egetctx{} effect.

% Example of how thread-local variables can be used for a tracing log.

Since all fibers of one scheduler execute concurrently on one system-level thread, they have exclusive access to the thread-local variables while they are running.
This allows a practical form of shared state without the overhead of synchronization primitives of multithreaded data structures.
Some example use-cases are per-scheduler debug tracing of events, where all fibers of one scheduler write to a common log,
and inter-fiber message passing, where fibers use a simple queue to exchange messages.
Of course, this comes with the restriction that it is only usable for fibers running in the same thread.

% What we want to verify about thread-local variables. 

In Eio thread-local variables are represented by a dictionary from variable names to arbitrary values and expose an intended API that only allows adding new entries.
However, to integrate thread-local variables into our model we simplify this into a single reference.
Properties we want to prove about our thread-local variable are:
\begin{enumerate}
  \item Each time a fiber performs a \egetctx{} effect it receives the same reference.
  \item As long as a fiber does not perform other effects like \efork{} or \esuspend{}, it holds exclusive ownership of the reference.
\end{enumerate}

Code examples illustrating the properties are shown in figure~\ref{fig:tlv-example}.
Note that these are only the most basic properties showing that \ocamlin{tlv} acts like a normal reference, but one that can be accessed via an effect.
To enable modular proofs of concrete fibers using the thread-local variable, we include in our logical definition a predicate \(\Omega\) on the stored value that can be instantiated by fibers as needed.

\begin{figure}[ht]
  \begin{minted}{ocaml}
let fiber1 () =
  let tlv1 = perform (GetContext ()) in
  (* ... *)
  let tlv2 = perform (GetContext ()) in
  assert (tlv1 = tlv2)

let fiber2 () =
  let tlv = perform (GetContext ()) in
  let v = !tlv in
  (* some computation that does not perform Fork/Suspend *)
  (* ... *)
  assert (!tlv = v)
  \end{minted}
  \caption{Constructed example of safety for thread-local variables.}
  \label{fig:tlv-example}
\end{figure}

\subsection{Changes to Logical State}

% How do we extend the protocol and change the logical definitions.
To handle a thread-local variable we must change both the source code and logical definitions.
The necessary changes to the implementation are trivial, as the scheduler creates a new reference as part of setting up the runtime environment and passes it to the continuation of the \egetctx{} handler as shown in figure~\ref{fig:sched-run-tlv}.
\begin{figure}[H]
  \begin{minted}[escapeinside=@@]{ocaml}
let run @\emph{\textbf{init}}@ (main: unit -> 'a) : 'a =
  (* ... *)
  let rec execute @\emph{\textbf{tlv}}@ fiber = 
    match fiber () with
    (* ... *)
    | @\textbf{effect}@ (GetContext ()), k ->  continue k @\emph{\textbf{tlv}}@
  in
  let @\emph{\textbf{tlv}}@ = ref init in
  execute @\emph{\textbf{tlv}}@ (fun () -> result := Some (main ())); 
  (* ... *)
  \end{minted}
  \caption{Changes to integrate a thread-local variable into the \ocamlin{Scheduler.run} code.}
  \label{fig:sched-run-tlv}
\end{figure}

The updated definitions are described in figure~\ref{fig:logical-state-ext}.
\(\gsFiberResources{l_{tlv}}{\Omega}\) represents all resources that a fiber owns while it is running, so each fiber specification now has this as a precondition.
The predicate at the moment only includes the points-to connective of \(l_{tlv}\) and that its value satisfies \(\Omega\).
Also, we must change the definition of \(\gsReadyF{}\) to include \(\gsFiberResources{l_{tlv}}{\Omega}\) in the pre- and postcondition so that each suspended fiber in the run queue gets access to the resources when it resumes execution.

\begin{figure}[ht]
  \begin{align*}
    \gsFiberResources{l_{tlv}}{\Omega} & \triangleq\; \exists v.\; l_{tlv} \mapsto v \ast \Omega\; v                                                    \\
    \gsReady{3}~l_{tlv}~\Omega~k       & \triangleq\; \gsFiberResources{l_{tlv}}{\Omega} \wand{} \ewp{k\; ()}{\bot}{\gsFiberResources{l_{tlv}}{\Omega}}
  \end{align*}
  \caption{Updated logical state definitions to model the thread-local variable.}
  \label{fig:logical-state-ext}
\end{figure}

The effect protocols of \efork{} and \esuspend{} are amended so that they pass the fiber resources from a fiber to the scheduler and from there to the next running fiber via the protocol pre- and postconditions as shown in figure~\ref{fig:coop-protocol-ext}.
A fiber uses the \egetctx{} effect to receive the concrete \(l_{tlv}\) reference, and by virtue of running it already holds the fiber resources to safely dereference \(l_{tlv}\).
The crux is that now the whole protocol \proto{3} is parameterized by the location and the predicate on its value.
We must do this instead of quantifying them at the effect-level so that the fiber and the scheduler agree on the concrete location.

\begin{figure}[ht]
  \begin{align*}
    \proto{3}~l_{tlv}~\Omega & \triangleq \begin{aligned}[t]
                                            Fork\;       & \#\; \begin{aligned}[t]
                             & !\; e\; (e)\; \begin{aligned}[t]
                            & \big\{ \gsFiberResources{l_{tlv}}{\Omega} \ast                                                                                             \\
                            & \later\; (\gsFiberResources{l_{tlv}}{\Omega} \wand{} \ewp{e\ \ ()}{\proto{3}~l_{tlv}~\Omega}{\gsFiberResources{l_{tlv}}{\Omega}}) \big\} .
                         \end{aligned} \\
                             & ?\; ()\; \{ \gsFiberResources{l_{tlv}}{\Omega} \}
                          \end{aligned}                                                 \\
                                            Suspend\;    & \#\; \begin{aligned}[t]
                             & !\; reg\; W\; S\; (reg)\; \{\gsFiberResources{l_{tlv}}{\Omega} \ast \gsIsReg{1}\; reg\; W\}. \\
                             & ?\; v\; (v)\; \{ \gsFiberResources{l_{tlv}}{\Omega} \ast W~v \}
                          \end{aligned} \\
                                            GetContext\; & \#\; !\; ()\; \{\top\}.\; ?\; (l_{tlv})\; \{ \top \}
                                          \end{aligned}
  \end{align*}
  \caption{Definition of extended \proto{3} protocol with \efork{}, \esuspend{}, and \egetctx{} effects.}
  \label{fig:coop-protocol-ext}
\end{figure}
%
These changes suffice to prove the safety of the two examples in figure~\ref{fig:tlv-example}.
In the next section we will show an example of how one can use the \(\Omega\) predicate to establish constraints on the thread-local variable.

