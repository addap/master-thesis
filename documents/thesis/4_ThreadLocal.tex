\section{Adding Support for Thread-Local Variables}
\label{sec:thread-local-vars}

% General information about the GetContext effect and that it's used to implement thread-local variables

So far we have looked at a protocol \(\proto{2}\) that has two effects which suffice to model fibers that can suspend and fork off new fibers.
But fibers in Eio can use an additional effect called \egetctx{} that we discuss in this section.
For each fiber the scheduler keeps track of context metadata, one part of which are \emph{thread-local variables}.
Thread-local variables are state that is shared between all fibers of one scheduler (hence thread-local) and a fiber gets access to them via the \egetctx{} effect.

% Example of how thread-local variables can be used for a tracing log.

Since all fibers of one scheduler execute concurrently on one system-level thread, they have exclusive access to the thread-local variables while they are running.
This allows a practical form of shared state without the overhead of synchronization primitives of multithreaded data structures.
Two example use-cases are per-scheduler tracing of events, where all fibers of one scheduler write to a common log,
and inter-fiber message passing, where fibers use a simple queue to exchange messages.
Of course, this comes with the restriction that it is only usable for fibers running in the same thread.

% What we want to verify about thread-local variables. 

In Eio thread-local variables are represented by a dictionary from variable names to arbitrary values and expose an intended API that only allows adding new entries.\todo{Correct that you can't actually change the whole dict. But we still just use an optional integer for simplicity.}
However, it is still possible for fibers to arbitrarily modify the whole dictionary, so for demonstration purposes we model thread-local variables as a single mutable reference that is part of the context record: \ocamlin{ctx.tlv}.
Properties we want to prove about thread-local variables are:
\begin{enumerate}
    \item Each time a fiber performs a \egetctx{} effect it receives the same reference.
    \item As long as a fiber does not perform other effects like \efork{} or \esuspend{}, it holds exclusive ownership of the reference.
\end{enumerate}

Code examples illustrating the properties are shown in figure~\ref{fig:tlv-example}.
Note that these are only the most basic properties showing that \ocamlin{ctx.tlv} acts like a normal reference, but one that can be accessed via an effect.
To enable modular proofs of concrete fibers using thread-local variables, we include in our logical definition a predicate \(T\) on the stored value that can be instantiated by fibers as needed.

\begin{figure}[ht]
    \begin{minted}{ocaml}
let fiber1 () =
  let tlv1 = perform (GetContext ()) in
  let tlv2 = perform (GetContext ()) in
  assert (tlv1 == tlv2)

let fiber2 () =
  let tlv = perform (GetContext ()) in
  let v = !tlv in
  (* some computation that does not perform Fork/Suspend *)
  ...
  assert (!tlv == v)
  \end{minted}
    \caption{Constructed example of safety for thread-local variables.}
    \label{fig:tlv-example}
\end{figure}

\subsection{Changes to Logical State}

% How do we extend the protocol and change the logical definitions.
To handle thread-local variables in our development we must change both the implementation and logical definitions.
The necessary changes to the implementation are trivial, so we just refer to the mechanization\footnote{TODO insert link}.
The new definitions are described in figure~\ref{fig:logical-state-ext}.
\(\gsTLVAg{}\; \delta\; tlv\) is used to show the uniqueness of the location \(tlv\).
\(\gsIsFiberContext{}\; \delta\; tlv\) represents the context that is tracked for each fiber, where \(\delta\) is a shorthand for multiple ghost names.
It expresses that the location \(tlv\) is a thread-local variable which maps to some value \(v\) satisfying \(T\).
The predicate T is hidden behind a \(\gsSavedPred{}\) indirection to make the mechanization easier.
\(\gsFiberResources{}\; \delta\) is then used to abstract away the concrete location \(tlv\).
This predicate represents all resources that a fiber owns while it is running, so each fiber specification now has this as a precondition.
Finally, we must change the definition of \(\gsReadyF{}\) to require \(\gsFiberResources{}\) as a precondition because it is needed to invoke the continuations saved in the scheduler's run-queue.

\begin{figure}[ht]
    \begin{align*}
        \gsTLVAg{}\; \delta\; tlv \triangleq\;            & \ownGhost{\delta}{agree(tlv)}  \qquad \textrm{Persistent}(\gsTLVAg{\delta}{tlv})                          \\
        \gsIsFiberContext{}\; \delta\; tlv\; \triangleq\; & \gsTLVAg{}\; \delta\; tlv \ast \exists T\; v.\; tlv \mapsto v \ast \gsSavedPred{}\; \delta\; T \ast T\; v \\
        \gsFiberResources{}\; \delta\; \triangleq\;       & \exists\; tlv.\; \gsIsFiberContext{}\; \delta\; tlv\;                                                     \\
        \gsReady{3}\; \delta\; f \triangleq\;              & \gsFiberResources{}\; \delta\; \wand{} \ewp{f\; ()}{\bot}{\top}
    \end{align*}
    \caption{Logical state definitions for the verification of our Eio model.}
    \label{fig:logical-state-ext}
\end{figure}

The effect protocols of \efork{} and \esuspend{} are amended so that they pass \(\gsFiberResources{}\) from a fiber to the scheduler and from there to the next running fiber via the protocol pre- and postconditions as shown in figure~\ref{fig:coop-protocol-ext}.
The \efork{} effect now also passes the concrete reference that should be used as the thread-local variable of the new fiber.
A fiber uses the \egetctx{} effect to receive the fiber context value and a copy of \(\gsTLVAg{}\).
This is used to show that the reference \(ctx.tlv\) is equal to the one from \(\gsFiberResources{}\) that the fiber already owns so that the contained points-to predicate can be used.

The crux is that now the protocol \proto{3} is parameterized by the ghost name \(\delta\) that identifies the thread-local variable.
This so that both the fiber and the scheduler agree on this ghost name.

\begin{figure}[ht]
    \begin{equation*}
        \proto{3}~\delta \triangleq \begin{aligned}[t]
            Fork\;       & \#\; \begin{aligned}[t]
                                     & !\; tlv\; e\; ((tlv, e))\; \begin{aligned}[t]
                                                   & \big\{ \gsFiberResources{}\; \delta\; T \ast \gsTLVAg{}\; \delta\; tlv\; \ast                              \\
                                                   & \later\; (\gsFiberResources{}\; \delta\; T \wand{} \ewp{e}{Coop}{\gsFiberResources{}\; \delta\; T}) \big\}
                                              \end{aligned} . \\
                                     & ?\; ()\; \{ \gsFiberResources{}\; \delta\; T \}
                                \end{aligned}                                 \\
            Suspend\;    & \#\; \begin{aligned}[t]
                                     & !\; reg\; W\; (reg)\; \{\gsFiberResources{}\; \delta\; T \ast \gsIsReg{1}\; reg\; W\}. \\
                                     & ?\; y\; (y)\; \{ \gsFiberResources{}\; \delta\; T \ast W\; y \}
                                \end{aligned} \\
            GetContext\; & \#\; !\; ()\; \{\top\}.\; ?\; ctx\; (ctx)\; \{ \gsTLVAg{}\; \delta\; ctx.tlv \}
        \end{aligned}
    \end{equation*}
    \caption{Definition of extended \proto{3} protocol with \efork{}, \esuspend{}, and \egetctx{} effects.}
    \label{fig:coop-protocol-ext}
\end{figure}

These changes suffice to prove the safety of the two examples in figure~\ref{fig:tlv-example}.

