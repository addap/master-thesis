\section{Extending the Scheduler with Thread-Local Variables (WIP)}
\label{sec:thread-local-vars}

\begin{itemize}
    \item How thread-local variables can be used.
    \item Explain the GetContext effect in Eio and how we model it in our scheduler.
    \item How we adapt our logical state to include GetContext.
    \item And explain that we need to parameterize the protocol to solve the issue of shared knowledge between the scheduler and fiber.
\end{itemize}

\begin{figure}[ht]
    \begin{alignat*}{2}
        Coop \delta \coloneqq\; &  & \quad Fork\; \#    & \; !\; e\; (e)\; \{\later \ewp{e}{Coop}{\top}\}. ?\; ()\; \{ \top \}        \\
                                &  & Suspend\;    \#    & \; !\; reg\; P\; (reg)\; \{isRegister\; reg\; P\}.?\; y\; (y)\; \{ P\; y \} \\
                                &  & GetContext\;    \# & \; !\; ()\; \{\top\}.?\; ctx\; (ctx)\; \{ isFiberContext\; \delta\; ctx \}
    \end{alignat*}
    \caption{Definition of \proto{} Protocol with \efork{} \& \esuspend{} Effects.}
    \label{fig:coop-protocol-ext}
\end{figure}

% Definition FORK_pre (Coop : pEff) (δ : gname): iEff Σ :=
%   >> (ℓstate : loc) e >> !(Fork' #ℓstate e) {{fiberState δ ℓstate ∗ ▷ (fiberContext δ -∗ EWP e #() <|Coop δ |> {{_, fiberContext δ}}) }};
%   << (_: val) << ?(#())        {{ fiberState δ ℓstate }} @ OS.
%     (* a.d. TODO I think I can delete promiseInv from SUSPEND *)

% Definition SUSPEND (δ : gname) : iEff Σ :=
%   >> (f: val) (P: val → iProp Σ) >> !(Suspend' f) {{
%     (* We call suspender with the waker function and waker receives a value satisfying P. *)
%       ( (∀ (waker: val),
%         (∀ (v: val), P v -∗  (EWP (waker v) <| ⊥ |> {{_, True }}) ) -∗
%         (▷ EWP (f waker) <| ⊥ |> {{_, True  }}) ) 
%     ∗
%       fiberContext δ)%I
%   }};
%   << y           << ?(y)         {{ P y ∗ fiberContext δ }} @ OS.

% Definition GET_CONTEXT (δ: gname ): iEff Σ :=
%   >> (_: val) >> !(GetContext') {{ True }};
%   << (state : loc) << ?(#state) 
%       {{isFiberState δ state}} @ OS.
